\subsection{Further Question 2}
\label{Further Question 2 SubSection}

\begin{tcolorbox}[colback=gray!20!white,colframe=gray!20!white]
  \emph{In the output activation equation below, what would the consequence be of replacing the +1 term with +10? How could one obtain a similar effect by changing a parameter of this equation? [Mark: 7]}

\begin{equation}
    y_j = \frac{\gamma[V_m - \Theta]_+}{\gamma[V_m - \Theta]_+ + 1}
\end{equation}
\end{tcolorbox} 
\vspace{0.5cm}

By plotting the rate code output function above and using $V_m$ as the unknown variable where $\gamma$ and $\Theta$ are fixed variables, it's seen that by increasing the denominator from +1 to +10, the output activation value ($y_j$) decreases in value as when compared to +1. 