\section{Conclusion}
\label{Conclusion Section}

Though both the fairness-aware algorithms are similar in the output goal, in which they process data into non discriminatory. However they are vastly different with the decision tree algorithm allowing for further usage and on larger more complex data-sets than that of the proposed 2 Naïve Bayes algorithm in \cite{Naive}. Such examples would include that the Naïve Bayes model is useful for gender discrimination whereas the decision tree would look at age as it will be able to create new branches base off of the splitting criterion. Gender seeks only male or female, the data is split into two models as $M_+$ and $M_-$, where age seeks not only up to 100 values but also groups of age brackets, to which can be branched off the tree where the entropy of the splitting criterion will justify the next node.