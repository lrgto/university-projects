\subsection{Magnox Mini Project Plan}
\label{Project Plan SubSection}

\subsubsection*{Project Outline:}
\label{Project Outline SubSubSection}

With nuclear waste within the UK on the rise, more nuclear waste is being made as the projection of power that is made by nuclear power plants is likely to increase in the next decade. The goal of this mini project in relation to Magnox nuclear waste is to design a cheap, strong, reliable and feasible container that will hold and transport nuclear waste. I will begin researching the type of nuclear waste the UK disposes of and how its currently transported and disposed of, then I will look at specific materials used to transport nuclear waste, why they are used and highlight positives and negatives of these methods. From this research I will begin designing and constructing the ideal container factoring in variables such as the intensity of radioactive waste against the type and size of the materials used in the container and what happens to the containers after transporting radioactive waste. By researching what radioactive waste is being disposed of and how it will get to the specified facility, I can specify what hazards the container may face without cracking, leaking or breaking the inner core that holds the radioactive waste. \\

As I cannot physically build and test such a container, I will gather data from scientific journals to show the penetration of gamma rays, x-rays, alpha and beta particles through lead. Through these values and my understanding of nuclear and particle physics I shall calculate numerous values for the tensile strength, weight, ray penetration (leakage), material degradation and other key factors. Ultimately, I shall have to prove mathematically that the container works. \\

My current theoretical proposal is a hexagonal container lined with lead core surrounded by a perpendicular titanium honeycomb inner wall, covered by aluminium or another composite material. Its proven that the thickness and quality of the lead wall greatly determines the penetration of gamma rays, which ultimately will affect the design of the container. The container should be as light weight as possible without prompting any hazards, the proposed titanium honeycomb wall shall improve the strength of the container and prevent deformation of the lead core and the aluminium / composite material will act protect the honeycomb structure from wear and tear, rust and other environmental hazards allowing the container to be built to last. 

\subsubsection*{Magnox Nuclear Waste Project Timeline:}
\label{Magnox Nuclear Waste Project Timeline SubSubSection}
\vspace{0.2cm}

\textbf{\underline{Week 10:}} \\ [0.2cm]
Initial mini project option selected. The construction of this project plan will act as a guide for this project and give weekly goals to achieve.\\ 

\textbf{\underline{Week 11:}} \\ [0.2cm]
Basic research completed, construct multiple ideas for a container to work off of. The submission and finalisation of the project plan. The project report to be started and the project report introduction to be wrote.\\

\textbf{\underline{Week 12:}} \\ [0.2cm]
Start research relating to the restrictions of the container such as materials, shapes, general design and thus a series of crude ideas and possible designs to be constructed. Thus, start the research in section the project report.\\

\textbf{\underline{Week 13:}} \\ [0.2cm]
Finish all research and start looking at a series of ideal materials for the construction of the container, calculations of the containment of radioactive material through gamma ray, x-ray, alpha and beta particles penetration of the material will be used to prove the most ideal material.\\

\textbf{\underline{Week 14:}} \\ [0.2cm]
Finish all research. Start finalising 2/3 designs and proving all positives and negatives and compare them. Proving through mathematical calculations of radioactive penetration and strength will be used to primarily judge which design is the best to pursue.\\

\textbf{\underline{Week 15:}} \\ [0.2cm]
Complete all current sections in the project report (Introduction, research, designs). Continues with calculation proving that the chosen design work and is suitable for the task its design to do and hazards it will face. Investigate errors and develop a report of these errors and ensure these errors are within suitable parameters (previously researched).\\

\textbf{\underline{Week 16:}} \\ [0.2cm]
Finish all calculations and finalise the design of the container allowing for the error analysis, complete the final design section in the project report include final calculations, schematics and errors.\\

\textbf{\underline{Week 17:}} \\ [0.2cm]
This week is allocated to allow as a buffer week in case of any problems encountered throughout the project. If not, problems are encounter then it shall be used as a pre-finalisation of my final report and check of error analysis.\\

\textbf{\underline{Week 18:}} \\ [0.2cm]
Finalise and check all research, calculations, errors, designs and report. Produce project analysis and conclusion. Submit project report.\\

