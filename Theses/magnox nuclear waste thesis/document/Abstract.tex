As the United Kingdom expands its nuclear power production, it will produce more radioactive waste and so research into its disposal is currently a big area. Though the use of current waste containers and advancement into increasing structural integrity of materials, a theoretical proposed design is produced in which accounts for current disadvantages posed by current container from radioactive waste collection, transport, storage and disposal and what types of waste is produced. Current materials and design features shape the final proposed design where safety and secondary containers influence its design, as to be single use container from the nuclear facilities to storage. The proposed design is designed to be stacked on top of each other to save space but to allow sufficient airflow and also have a replaceable exterior for removable of corrosion covering a honeycomb structure that improves the tensile strength of the design and explore carbon nanotubes which acts as a thin honeycomb layer and improves tensile strength and thermal conductivity. Errors arise due compact heat generation and radiation levels when stacked. Theoretical the proposed design is an acceptable solution to a future problem as the U.K’s radioactive waste grows as to store the waste in compact areas.