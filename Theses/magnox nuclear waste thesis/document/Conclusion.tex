In conclusion to the project brief outlined in \cref{Introduction Section}, current containers were analysed and the types of radioactive wastes that are handled through collection at nuclear facilities to transport to storage/ disposal and recycling/ reprocessing. Advantages and disadvantages were drafted per container that's currently in use with respect to the life cycle of the containers and an initial design was drafted to combat some disadvantages currently impacting radioactive waste disposal. Future issues such as space consumption for a projected increase in radioactive waste were also taken into account in the proposed designs. \\

The proposed design fixes space consumption issues while allowing for sufficient airflow allowing for the built up gases to escape the container and also allow for the containers to be air cooled, which matches the current 500L drum in \cref{500 Litre Drum SubSubSection} but takes up much less space. The design has the ability to be customised in physical size and also the option for cement, lead lined walls and side wall plate replacements while not changing the physical size of the container. The ability to replace the side panels without exposing the radioactive materials allow for corrosion and materials weaknesses to be removed and replaced without having to put the container into a larger container much like the 500L drum being placed into a concrete box or a tru-shield container thus saving more space. \\

The use of a honeycomb structure to increase the tensile strength of a weak material such as aluminium where is can absorb external damage and easily replaced.The improvement of filling the holes in the honeycomb structure with cement improves it strength even further, the concrete box has a steel bar frame and concrete walls where it can become brittle or break where as an aluminium honeycomb structure will improve the tensile strength of the wall. The possibly use of the carbon nanotube to on the multiple surfaces would improve the tensile strength much like a honeycomb structure whereas carbon nanotubes and smaller allowing for space saving containers to gain further strength.\\

The proposed design theoretically works though requires further testing and analysis, the idea of a space saving container which can have parts replaced will continue saving space and saving money as placing small containers into larger one will no longer be required.