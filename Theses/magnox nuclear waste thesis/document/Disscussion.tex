Though this project is 100\% theoretical due to the restrictions upon testing the design in a real world situation, small laboratory experiments specific to the proposed design would have proven insightful and given clarity on the success or failure of the design and or shown areas of improvements needed. The design is based off of turning disadvantages of current containers into advantages where by highlighting design improvements or developing solutions to long term issues such as storage space. The United Kingdom has pledged a growth in nuclear power that will result in more radioactive waste as outlined in the project brief in \cref{Introduction Section}, where all radioactive waste will take years to be at safe levels until ready for regular disposal. \\

The issues raised in \cref{Nuclear Waste in the United Kingdom SubSection} show that that nuclear waste must be kept away from the public due to its hazardous nature. Another problem arises where the United Kingdom doesn't have much land space especially if a container breaks or a storage facility collapses causing the landscape to become radioactive. This forces the design of these containers to consider space saving methods such as stacking containers on top of one another. The proposed design highlights this issue and bring a simple geometry solution much like the concrete boxes supply in \cref{Concrete Boxes SubSubSection}. \\

Further analysis of the proposed design could have been worked through a series of laboratory experiments measuring different levels of radiation penetration through different materials such as lead, aluminium, titanium and steel where these materials are placed in a honeycomb structure. Outline in \cref{Data Analysis Section}, a decaying isotope of Cesium 137 was used to measure such particle penetration and proved valuable in supplying evidence that lead is a good material for radioactive shielding. These measurements though inconsistent with the project outline due to the safety issues of radiation levels could have proven useful in determining how the honeycomb structure works with radiation penetration. Other experiments such to measure the tensile strength of a steel, aluminium and potentially titanium honeycomb structure to show which worked better while keeping weigh low and what's the ideal thicknesses of the walls of the honeycomb structure. \\
\newpage
In summary, research specific into what exactly the U.K. does and the designs of specific containers were vague and so logical physical reasoning from other similar containers were sourced. The total sum of the project went highly well as outline in \cref{Conclusion Section} where current problems were isolated and addressed, future problems were derived through the statement that the U.K. is planning on expanding its nuclear power presence which will ultimately produce more radioactive waste where when dealing with radioactivity is still new and due to the hazard of radiation the waste must be stored for years if not generations. The brief was too design a container that could transport ILW and HLW materials however the question arises on how the U.K. can reduce it's radioactive waste as it expands its nuclear power stance. This can be done through limiting material exposure and further research into radiation retardant materials and clothing, though radiation truly limits what the mankind can do to limit, reduce and dispose of its radioactive waste.