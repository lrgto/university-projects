The United Kingdom currently operates a series of nuclear reactors generating around 19\%-20\% of its overall electricity \cite{Percentage} though some of these reactors are coming to the end of their safe operational life the U.K. plans to expands and increase its electricity generated by these reactors. Nuclear reactors although hazardous harness the power of the atom to extract a vast amount electricity over a substantial long period while consuming low quantities of fuel, however nuclear reactors produce highly radioactive waste and disposing of this waste is equally as hazardous. However nuclear reactors are not the only source of radioactive material, organisations and businesses also produce radioactive waste and so the U.K. must transport and dispose of this waste correctly and safely. In transport the waste is placed into special containers that are designed to hold hazardous materials however research into more secure and affordable containers is an ongoing endeavour as outlined in this project's brief; \\

\vspace{-1.2cm}
\subsection*{ }
\label{Project Brief SubSection}
\begin{minipage}{1\textwidth}
\begin{center} \emph
"The project is based on the disposability of radioactive waste from Nuclear Power Stations. Specifically the design of a suitably shielded robust waste container for intermediate and higher level active wastes that could be transported by road and rail to the UK Geological Disposal Facility. The project is to propose, design, model and substantiate a novel design for such a container using nuclear physics knowledge. Nuclear power in the United Kingdom generates around 19\% of the country’s electricity as of 2020, projected to rise to a third by 2035 and the disposability of radioactive waste is a big area of research in the UK currently." \cite{MiniProjectOptions} 
\end{center}
\end{minipage}
\vspace{0.5cm}

Due to construction restrictions, a theoretical design must be formed off of the handling of current nuclear waste in the U.K, in transport and disposal. Such a design will be new or obtain improvements of current containers currently in use, following the brief where the design is to conform to the standards of carrying intermediate and higher levels of nuclear waste from nuclear power plants. However some nuclear waste from these facilities such as spent fuel rods are deemed to valuable to be classed as waste so they undergo treatment through reprocessing and reusing the fuel \cite{ManageNuclearWaste}. Thus the proposed design should transport all forms of radioactive materials and those deemed as waste can be stored in the same proposed design, which limits the hazard to human life and the environment but also is cost effective. The proposed design is to be based off of current designs, focused heavily on turning the disadvantages of the current containers into advantages where the proposed model shows a solution to a future problem. The proposed design is formed to fit the current regulations and method of radioactive material disposal currently adopted by the United Kingdom.